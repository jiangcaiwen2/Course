\documentclass{article}
\usepackage{graphicx,fancyhdr,amsmath,amssymb,amsthm,subfig,url,hyperref}
\usepackage[margin=1in]{geometry}
\usepackage{enumerate}
% \usepackage{algorithms}
\usepackage{bookmark}
\usepackage{tikz}
% \usetikzlibrary{arrows}
\usetikzlibrary{arrows.meta}
\usetikzlibrary{quotes}
\usepackage{extarrows}
\usepackage{clrscode3e}

%----------------------- Macros and Definitions --------------------------

%%% FILL THIS OUT
\newcommand{\studentname}{Shen Linfeng}
\newcommand{\suid}{2018E8018461028}
\newcommand{\exerciseset}{Homework 4}
\newcommand{\floor}[1]{\left\lfloor #1 \right\rfloor}
\newcommand{\ra}{\rightarrow}
\def\o{\ensuremath\varnothing}
%%% END



\renewcommand{\theenumi}{\bf \Alph{enumi}}

%\theoremstyle{plain}
%\newtheorem{theorem}{Theorem}
%\newtheorem{lemma}[theorem]{Lemma}

\fancypagestyle{plain}{}
\pagestyle{fancy}
\fancyhf{}
\fancyhead[R]{\sffamily\bfseries ShanghaiTech University}
\fancyhead[L]{\sffamily\bfseries CS240 Algorithm Design and Analysis}
\fancyfoot[L]{\sffamily\bfseries\studentname: shenlinfeng2018@sari.ac.cn}
\fancyfoot[R]{\sffamily\bfseries\thepage}
\renewcommand{\headrulewidth}{1pt}
\renewcommand{\footrulewidth}{1pt}

\graphicspath{{figures/}}

%-------------------------------- Title ----------------------------------

\title{CS240 \exerciseset}
\author{\studentname \qquad Student ID: \suid}

%--------------------------------- Text ----------------------------------

\begin{document}
\maketitle

\section*{Problem 1}
\begin{enumerate}[1)]
    \item  Given an assignment in which at most $k$ variables are true, we can calculate the formula in polynomial time,
    so this problem is in NP.
    \item 
    Consider the SAT : Given CNF formula $\Phi $, does it have a satisfying truth assignment?

    Let $k$ be the number of input variables of $\Phi$, 
    then (can $\Phi$ be satisfied by an assignment in which at most $k$ variables are true) is an instance of the origin problem.
    \item 
    \begin{itemize}
        \item If we have yes to this SAT problem, which mean we can find an assignment with $k$ variables to make $\Phi$ true. And in this case, the number of variables is true will be $\leqslant k$, so we can get yes to the origin problem.
        \item If we have yes to the origin problem, which mean we can find an assign to make $\Phi$ true, so we can get yes to this SAT problem.
    \end{itemize}
    
\end{enumerate}
Thus this problem is at least as hard as SAT problem.
Since the SAT problem is NP-complete, 
we have this problem is also NP-complete.



\pagebreak
\section*{Problem 2}
\begin{enumerate}[1)]
    \item  Given a simple cycle in $G$, we can determine whether the sum of its  edge weights is zero in polynomial time.  
    Thus Zero-Weight-Cycle is in NP.
    % Prove that problem A is in NP.
    \item Consider the Directed-Hamiltonian-Cycle problem $G=(V,E)$. 
    Let $n=|V|$.
    Construct a weighted directed graph $G'$ with $2n$ nodes, 
    such that every $v_i \in V$ corresponds to two nodes, $u_i, w_i \in V'$.

    Add an edge from $u_i$ to $w_i$ with weight $1$ for every $i<n$, 
    and add an edge from $u_n$ to $w_n$ with weight $(1-n)$. 
    
    And then add edges $(w_i, u_j)$ with weight $0$ for every edge $(v_i, v_j)\in E$.

    So, (is there a simple cycle in $G'$ so that the sum of the edge weights on this cycle is exactly $0$) is an instance of Zero-Weight-Cycle Problem.
    \item 
    When there is no edge from $w_i$ to $w_j$, or $u_i$ to $u_j$ in $G$, 
    a cycle of $G$ must be $u_a,w_a,u_b,w_b,\cdots,w_e,u_a$. 
    With weight of every $(w_i,u_j)$ is $0$, and weight of $(u_i,w_i)$ is positive for $i<n$. 
    So a simple cycle of the sum-weight is $0$ iff the cycle contain all edge $(u_i,w_i)$ in $G'$, which mean it contain all nodes of $G'$'s.
    \begin{itemize}
    \item So if we have yes to this Zero-Weight-Cycle problem, 
        we can find a zero-weight-cycle in $G'$, 
        then this cycle must a simple cycle contain all nodes in $G'$, and we can replace $u_i$ and $w_i$ with $v_i$ to get the solution of Directed-Hamiltonian-Cycle, which mean we get yes to this Directed-Hamiltonian-Cycle problem.

    \item  On the other hand, if we have yes to this Directed-Hamiltonian-Cycle     problem,
        we can get a simple cycle that contain all nodes in $G$,
        then we can replace $v_i$ with $u_i$ and $w_i$ to get a zero-weight-cycle,
        which mean we get yes to this zero-Weight-Cycle problem.
    \end{itemize}
    % Prove that the yes/no answers to the two instances are the same.
\end{enumerate}

Thus this problem is at least as hard as Directed-Hamiltonian-Cycle problem.
Since the Directed-Hamiltonian-Cycle problem is NP-complete, 
we have Zero-Weight-Cycle is also NP-complete.

\pagebreak
\section*{Problem 3}
\begin{enumerate}[1)]
    \item  Given a one-to-one mapping $f: V'\rightarrow V$, we can check if every edge $(v_i, v_j)\in E'$ have $(f(v_i),f(v_j)) \in E$ in $O(|E|)$.
    Thus this problem is in NP.
    \item Consider the Independent Set problem $G = (V, E)$ with $k$.
    Construct another graph $G'=(V',E')$, with $V'$ is a set contain $k$ nodes, and $E'=\o$.
    
    So, (is $G'$ equivalent to a subgraph of $G$) is an instance of origin problem. 
    \item 
    \begin{itemize}
        \item If $G'$ is equivalent to a subgraph of $G$, then let $f: V'\rightarrow V$ be the map function. 
        Because $E'=\o$, we can know there is no edge between $u$ and $v$ for any $u,v\in V$, so there is no edge between $f(u)$ and $f(v)$. That mean the set $\{f(v)|~v\in V'\}$ is an independent set of $G$, 
        there is an independent set of size $\geqslant k$.

        \item On the other hand, if there is an independent set of size $\geqslant k$, 
        then select $k$ of them to a set $A$.
        It will be easy to map nodes from $A$ to $V'$ one-to-one.
        Because there is no edge between any two nodes from $A$, 
        so $G'$ is a subgraph of $G$.
    \end{itemize}
\end{enumerate}

Thus this problem is at least as hard as Independent Set problem.
Since the Independent Set problem is NP-complete, 
we have this problem is also NP-complete.

\pagebreak
\section*{Problem 4}
\begin{enumerate}[1)]
    \item Given a mapping as schedule $f: C\rightarrow S$,
    we can calculate the number of conflicts occurs for every $r\in R$,
    and then check if it bigger than $K$.
    It will cost $O(|C|\cdot |R|)$.
    Thus this problem is in NP.
    \item Consider the 3-Colorability problem: Given an undirected graph $G=(V,E)$ does there exist a way to color the nodes red, green, and blue so that no adjacent nodes have the same color?
    
    Construct set $C=V$ , $S={\text{red}, \text{blue}, \text{green}}$, 
    $R=\{\{u,v\}|~(u,v)\in E\} $.
    Let $K=0$, and it will be an instance of the sehedule problem with input $\{C,S,R,K\}$.
    \item In this situation, 
    \begin{itemize}
        \item If we get yes to this schedule problem, which mean we have a mapping $f$ to make the number of conflicts no more than $K=0$, 
        which mean for every $\{u,v\}\in R$, $f(u)$ and $f(v)$ are different.
        Then we can use $f$ to color $G$ to make no adjacent nodes have the same
        color.
        \item If we get yes to this 3-Colorability problem, which mean we have a mapping $f': V\rightarrow S$ to make no adjacent nodes have the same color. 
        Then we can use $f'$ to assign $C$ to $S$ to make sure there will no $\{u,v\}\in R$ will conflict.
    \end{itemize}
\end{enumerate}

Thus this problem is at least as hard as 3-Colorability problem.
Since the 3-Colorability problem is NP-complete, 
we have this problem is also NP-complete.


\pagebreak
\section*{Problem 5}
\begin{enumerate}[1)]
    % This problem can be convert to find a 
    \item Given a simple TA cycle, we can calculate the sum of nodes that the cycle contain and determine whether it $\geqslant K$ in $O(n)$.  
    Thus this problem is in NP.
    \item Consider the Longest Path problem: Given a digraph $G = (V, E)$, does there exists a simple path of length at least $k$ edges.
    
    Construct a digraph $G'=(V',E')$, 
    with $V'=V\cup \{s\}$, and $E'=E\cup \{(s,v)|~v\in V\}\cup \{(v,s)|~v\in V\}$. 

    Let every nodes in $V'$ represent a TA, $(v_i, v_j)$ mean $v_i$ works as a TA for $v_j$, and $K=k+2$.

    So, (is there a simple TA cycle containing at least $K$ TAs on $G'$) is an instance of the origin problem.
    \item 
    For every path in $G$ with $a$ nodes, which mean its length is $a-1$, can add $s$ to form a cycle in $G'$ with $a+1$ nodes.
    For every cycle in $G'$ with $a$ nodes, can remove $s$ and related edges to get a path with $a-1$ nodes and $a-2$ length in $G$ if it contain $s$, 
    or remove one node and and related edges to get a pathwith $a-1$ nodes and $a-2$ length in $G$ if it don't contain $s$. 

    \begin{itemize}
        \item 
    So, if we find a simple TA cycle containing at least $K$ TAs, 
    we can get a simple path in $G$ whose length is at least $K-2=k$. 
    This is the solution of the Longest Path problem.

    \item 
    On the other hand, if we find a simple path $A'$ in $G$ whose length is at least $k$, we can get  $A'\cup \{s\}$ could be a simple cycle whose length is at least $k+2=K$.
    \end{itemize}
\end{enumerate}

Thus this problem is at least as hard as Longest Path problem.
Since the Longest Path problem is NP-complete, 
we have this problem is also NP-complete.


\pagebreak
\section*{Problem 6}
\begin{enumerate}[1)]
    \item Given an input set $S$, we can check if the total weight is at most $b$ and if the corresponding profit is at least $k$. 
    It will cost $O(|S|)$.
    \item Consider the Subset Sum problem: Given natural numbers set $W =\{w_1 , \cdots, w_n\}$ and an integer $s$, is there a subset $S$ that adds up to exactly $s$.
    
    Let $n=|W|$, $k=b=s$, construct $n$ elements' set $A={a_1,a_2,\cdots,a_n}$ where $a_i=w_i$ for every $i$, 
    represent weighted of $i$th item, 
    $C={c_1,c_2,\cdots,c_n}$ where $c_i=w_i$ for every $i$, 
    represent profit of $i$th item.
    Then (find a subset of the items with total weight at most $b$ such that the corresponding profit is at least $k$) is an instance of the Knapsack problem.
    
    \item 
    \begin{itemize}
        \item 
    If we have yes to this Knapsack problem, there must have a set $S$ that:
    \begin{gather*}
        \sum_{i\in S}a_i \leqslant b \\
        \sum_{i\in S}c_i \geqslant k \\
    \end{gather*}
    Due to $a_i=c_i=w_i$ for every $i$, and $k=b=s$, we can get:
    \begin{gather*}
        \sum_{i\in S}w_i \leqslant b \\
        \sum_{i\in S}w_i \geqslant k \\
    \end{gather*}
    which mean:
    \begin{gather*}
        \sum_{i\in S}w_i = s \\
    \end{gather*}
    So, we also have yes to this Subset Sum problem.

    \item
    On the other hand, if we have yes to this Subset Sum problem, 
    which mean there is a set $S$ that:
    \begin{gather*}
        \sum_{i\in S}w_i = s;
    \end{gather*}
    So we can get:
    \begin{gather*}
        \sum_{i\in S}a_i =\sum_{i\in S}w_i=s=b \leqslant b \\
        \sum_{i\in S}c_i =\sum_{i\in S}w_i=s=k \geqslant k \\
    \end{gather*}
    So, we also have yes to this Knapsack problem.
    \end{itemize}
\end{enumerate}

Thus this problem is at least as hard as Subset Sum problem.
Since the Subset Sum problem is NP-complete, 
we have this problem is also NP-complete.

\end{document}
