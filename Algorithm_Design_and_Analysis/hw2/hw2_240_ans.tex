\documentclass{article}
\usepackage{graphicx,fancyhdr,amsmath,amssymb,amsthm,subfig,url,hyperref}
\usepackage[margin=1in]{geometry}
\usepackage{enumerate}
% \usepackage{algorithms}
\usepackage{extarrows}
\usepackage{clrscode3e}

%----------------------- Macros and Definitions --------------------------

%%% FILL THIS OUT
\newcommand{\studentname}{Shen Linfeng}
\newcommand{\suid}{2018E8018461028}
\newcommand{\exerciseset}{Homework 2}
\newcommand{\floor}[1]{\left\lfloor #1 \right\rfloor}
%%% END



\renewcommand{\theenumi}{\bf \Alph{enumi}}

%\theoremstyle{plain}
%\newtheorem{theorem}{Theorem}
%\newtheorem{lemma}[theorem]{Lemma}

\fancypagestyle{plain}{}
\pagestyle{fancy}
\fancyhf{}
\fancyhead[R]{\sffamily\bfseries ShanghaiTech University}
\fancyhead[L]{\sffamily\bfseries CS240 Algorithm Design and Analysis}
\fancyfoot[L]{\sffamily\bfseries\studentname: shenlinfeng2018@sari.ac.cn}
\fancyfoot[R]{\sffamily\bfseries\thepage}
\renewcommand{\headrulewidth}{1pt}
\renewcommand{\footrulewidth}{1pt}

\graphicspath{{figures/}}

%-------------------------------- Title ----------------------------------

\title{CS240 \exerciseset}
\author{\studentname \qquad Student ID: \suid}

%--------------------------------- Text ----------------------------------

\begin{document}
\maketitle

\section*{Problem 1}

\begin{enumerate}[(a)]
\item %(a)
Before you buy your $i$th toy, if you have exactly $j$ toy types, 
the probability of the $i$ toy is the type you already have is $\frac{j}{n}$;
before you buy your $i$th toy, if you have exactly $j-1$ toy types, 
the probability of the $i$th toy is the type you have not had is $\frac{n-(j-1)}{n}=\frac{n-j+1}{n}$.

So, we have 
\begin{equation*}
    p_{i,j}=\frac{j}{n}p_{i-1,j}+\frac{n-j+1}{n}p_{i-1,j-1}    
\end{equation*}

\item %(b)
Suppose you only have 0 toys before the first purchase.
Then we have some base case:
\begin{align*}
    p_{1,1} &= 1\\
    p_{i,j} &= 0 \ , i>j
\end{align*}

Direct use of recursion from will result in $O(2^n)$, 
but with saving past $p$ values we can calculate $p_{i,j}$ in $O(ij)$.
The algorithm is shown below:
\begin{codebox}
\Procname{$\proc{Toy}(i,j,n)$}
\li Initialize a $i\times j$ array $p$ with $0$, let $p[1][1]=1$
\li \If $i > j$
\li \Then \Return $0$
    \End
\li \For $a=1$ \To $i$
\li \Do
    \For $b=1$ \To $i$
\li \Do $p[a][b]=\frac{b}{n}p[a-1][b]+\frac{n-b+1}{n}p[a-1][b-1]$
\li     \If $b\geqslant j$
\li     \Then \textbf{break}
        \End
    \End
    \End
\li \Return $p[i][j]$
\end{codebox}
\end{enumerate}
\pagebreak

\section*{Problem 2}
Let $v$, $w$ be the arrays store value and weight, and $W$ is the capacity.
\begin{codebox}
\Procname{$\proc{Knapsack}(n,v,w,W)$}
\li $\id{v-sum}=\sum_{i=1}^n v[i]$
\li Initialize a $(n+1)\times (\id{v-sum}+1)$ array $\id{opt}$ with $\infty$
(for clearly, let the index begin from $0$ here).
\li \For $i\gets 1$ \To $n$
\li \Do \For $j\gets 1 \To \id{v-sum}$
\li     \Do \If $j\geqslant v[i]$
\li         \Then
                $\id{opt}[i][j]\gets \min\{\id{opt}[i-1][j], \id{opt}[i-1][j-v[i]]+w[i]\}$
\li         \Else
\li             $\id{opt}[i][j]\gets \id{opt}[i-1][j]$
            \End                 
        \End
    \End
\zi
\li Initialize an array $c[n]$
\li using Binary Search by the index $j$ of $\id{opt[i][j]}$ for every $i$, to find $c[i]=\max\{j|\ \id{opt}[i][j]\leqslant W\}$
\li \Return $\max\{c[i]\}$
\end{codebox}
The two for loop will cost $O(n\id{v-sum})$, 
and $n$ Binary Search will cost $O(n\log\id{v-sum})$.
From $v_i<n$ we can know $\id{v-sum}<n^2$, 
so this algorithm will run in $O(n^3)$.

~\\
\noindent
$\id{opt}[i][j]$ is the min weight subset of items $1,\cdots,i$ with value $j$.
There exist two case:
\begin{itemize}
    \item $\id{opt}$ does not select item $i$.\\
        \id{opt} selects best of $\{ 1, 2,\cdots, i-1 \}$ using value limit $j$.
    \item \id{opt} selects item $i$.\\
        new value limit = $j-v[i]$. \\
        \id{opt} selects best of $\{ 1, 2,\cdots , i–1 \}$ using this new value limit.
\end{itemize}
The rest of the analysis is similar to the original Knapsack problem, which can prove the correctness.

\pagebreak
\section*{Problem 3}
Let's create a data structure with: $S[i].x=x_i$, $S[i].y=y_i$.
\begin{codebox}
\Procname{$\proc{Counting-Friends}(S, n)$}
\li Sort $S$ in ascending order by $S[i].x$, it will cost $O(n\log n)$.
\li \Return $\proc{Count}(S, n)$ 
\end{codebox}

\begin{codebox}
\Procname{$\proc{Count}(S, n)$}
\li \If $n\isequal 1$
\li \Then \Return $0$
    \End
\zi
\li $\id{mid} \gets \floor{n/2}$
\li $\id{count}\gets 0$, $i \gets j \gets k \gets 1$
\li $\id{S-left} \gets S[1:\id{mid}]$
\li $\id{S-right} \gets S[\id{mid}+1:n]$
\li $\id{left} \gets \proc{Count}(\id{S-left}, \id{mid})$
\li $\id{right} \gets \proc{Count}(\id{S-right}, n-\id{mid})$
\zi
\li \While $i \leqslant \id{mid}$ \textbf{and} $j \leqslant n-\id{mid}$
\li \Do \If $\id{S-left}[i].y>\id{S-right}[j].y$
\li     \Then $\id{count}+=\id{mid}-i+1$
\li         $S[k]=\id{S-right}[j]$
\li         $j++$
\li     \Else 
\li         $S[k]=\id{S-left}[i]$
\li         $i++$
        \End
\li     $k++$
    \End
\li \If $i\leqslant \id{mid}$
\li \Then $S[k:n]=S[i:\id{mid}]$
    \End
\zi
\li \Return $\id{left}+\id{right}+\id{count}$
\end{codebox}
This problem can be convert to: when $S$ be sorted by $S.x$, 
then the number of Reverse Pairs is the answer. 
So it can be solved by Merge Sort.

~\\
\noindent
This algorithm modifies from Merge Sort, still run in $O(n\log n)$.

\pagebreak
\section*{Problem 4}
Let array $C$ store the cards, \func{Equ} be the equivalence tester.
\begin{codebox}
\Procname{$\proc{Equivalent-Detection}(C,n)$}
\li $\id{tmp} \gets C[1]$
\li $\id{num} \gets 1$
\li \For $i=2$ \To $n$
\li \Do \If $\func{Equ}(\id{tmp}, C[i])$
\li     \Then $\id{num}++$
\li     \Else
\li         $\id{num}--$
\li         \If $\id{num}\isequal 0$
\li         \Then $\id{tmp}\gets C[i]$
\li             $\id{num}\gets 1$
            \End
        \End
    \End
\li $\id{num}\gets 0$
\li \For $i=1$ \To $n$
\li \Do \If $\func{Equ}(\id{tmp}, C[i])$
\li     \Then $\id{num}++$
        \End
    \End
\li \If $\id{num}> n/2$
\li \Then \Return True
\li \Else
\li \Return False
\end{codebox}
Both two for loop run in $O(n)$, so this algorithm will run in $O(n)$.\\

\noindent
If there exist than $n/2$ cards that are all equivalent to one another, let the number of those cards be $n_1$. 
When in the first for loop and $\id{tmp}$ equivalent to those cards, 
assume it past $n_2$ other type cards, then we have $n_2\leqslant n-n_1<n/1$.
So we have $n_1-n_2>0$, which mean after the first for loop end, 
if there exist than $n/2$ cards that are all equivalent to one another
the $tmp$ is what we want(If exist). 
And the second for loop will check if it is true.

\pagebreak
\section*{Problem 5}
Let $A$,$B$,$C$ be the three sequences, 
and $m$, $n$ be the length of $A$ and $B$.
\begin{codebox}
\Procname{$\proc{Sequence-Merging}(A,B,C,m,n)$}
\li Initialize a $(m+1)\times (n+1)$ bool array $T$
\li \For $i\gets 0 \To m$
\li \Do \For $j\gets 0 \To n$
\li     \Do \If $i\isequal 0$ \textbf{and} $j\isequal 0$
\li         \Then $T[i][j]=\text{True}$
\li         \ElseIf $i\isequal 0$ \textbf{and} $B[j]\isequal C[j] $
\li         \Then $T[i][j]=T[i][j-1]$
\li         \ElseIf $j\isequal 0$ \textbf{and} $A[i]\isequal C[j] $
\li         \Then $T[i][j]=T[i-1][j]$
\li         \ElseIf $A[i]\isequal B[j]\isequal C[i+j]$
\li         \Then $T[i][j]=T[i-1][j]\ ||\ T[i][j-1]$
\li         \ElseIf $A[i]\isequal C[i+j]$
\li         \Then $T[i][j]=T[i-1][j]$
\li         \ElseIf $B[i]\isequal C[i+j]$
\li         \Then $T[i][j]=T[i][j-1]$
\li         \Else
\li             $T[i][j]=\text{{False}}$
            \End
        \End
    \End
\li \Return $T[m][n]$
\end{codebox}
Two for loop will cost $O(mn)$, so this algorithm will run in $O(mn)$.\\

\noindent
$T[i][j]$ store the possibility of if the first $i$ of $A$ and
the first $j$ of $B$ can compose the first $i+j$ of $C$.
All situations are considered in the for loop, so this algorithm will run well.

\pagebreak
\section*{Problem 6}
\begin{codebox}
\Procname{$\proc{Polynomial Multiplication}(A, B, n)$}
\li Initialize a $2^k$ array $C$ with $0$($k=\min\{k|\ 2^k>n,k\in \mathbb{Z}\}$).
\li \For $i=0$ \To $2^k-1$
\li \Do \For $p=0$ \To $n-1$
\li     \Do $C[i] = C[i]+A[p]\cdot B[p\oplus i]$
        \End
    \End
\li \Return $C$
\end{codebox}

The correctness can be proved by the definition and the fact of $p\oplus(p\oplus i)=i$.
Unfortunately,this is a brute force, will cost $O(n^2)$.

\end{document}
