\documentclass[12pt]{article}%
\usepackage{amsfonts}
\usepackage{fancyhdr}
\usepackage{comment}
\usepackage[a4paper, top=2.5cm, bottom=2.5cm, left=2.2cm, right=2.2cm]%
{geometry}
\usepackage{times}
\usepackage{amsmath}
\usepackage{changepage}
\usepackage{amssymb}
\usepackage{graphicx}%
\setcounter{MaxMatrixCols}{30}
\newtheorem{theorem}{Theorem}
\newtheorem{acknowledgement}[theorem]{Acknowledgement}
\newtheorem{algorithm}[theorem]{Algorithm}
\newtheorem{axiom}{Axiom}
\newtheorem{case}[theorem]{Case}
\newtheorem{claim}[theorem]{Claim}
\newtheorem{conclusion}[theorem]{Conclusion}
\newtheorem{condition}[theorem]{Condition}
\newtheorem{conjecture}[theorem]{Conjecture}
\newtheorem{corollary}[theorem]{Corollary}
\newtheorem{criterion}[theorem]{Criterion}
\newtheorem{definition}[theorem]{Definition}
\newtheorem{example}[theorem]{Example}
\newtheorem{exercise}[theorem]{Exercise}
\newtheorem{lemma}[theorem]{Lemma}
\newtheorem{notation}[theorem]{Notation}
\newtheorem{problem}[theorem]{Problem}
\newtheorem{proposition}[theorem]{Proposition}
\newtheorem{remark}[theorem]{Remark}
\newtheorem{solution}[theorem]{Solution}
\newtheorem{summary}[theorem]{Summary}
\newenvironment{proof}[1][Proof]{\textbf{#1.} }{\ \rule{0.5em}{0.5em}}

\newcommand{\Q}{\mathbb{Q}}
\newcommand{\R}{\mathbb{R}}
\newcommand{\C}{\mathbb{C}}
\newcommand{\Z}{\mathbb{Z}}

\begin{document}

\title{CS280 Fall 2019 Assignment 1 \\ Part A}
\author{Basic Neural Networks}
\date{Due on October 10, 2019, 23:59 UTC+8}
\maketitle

\paragraph{Name:}

\paragraph{Student ID:}

\newpage


\subsubsection*{1. Hessian in Logistic Regression (10 points)}
Let $\sigma(a)=\frac{1}{1+e^{-a}}$ be an activation function, the loss function of LR is  \[f(\mathbf{w})=-\sum_{i=1}^{n}[y_i\log(\mu_i)+(1-y_i)\log(1-\mu_i)],\] where $\mu_i=\sigma(\mathbf{w}^\intercal\mathbf{x}_i)$. 
\begin{itemize}
	\item Show that the Hessian of $f$ can be written as $H=X^\intercal S X$, where $S=diag(\mu_1(1-\mu_1),\cdots, \mu_n(1-\mu_n))$ and $X = [X_1,\cdots, X_n]^\intercal$
\end{itemize}


\newpage


\subsubsection*{2. Linear Regression (5 points)}
Linear regression has the form \[f(x)=E[y|x] = b + \mathbf{w}^\intercal \mathbf{x}.\]  
\begin{itemize}
	\item It is possible to solve for $\mathbf{w}$ and $b$ separately. Show that 
	\[b = \frac{1}{n}\sum_{i=1}^n y_i -\frac{1}{n}\sum_{i=1}^n \mathbf{x}_i^\intercal \mathbf{w} = \bar{y} - \bar{\mathbf{x}}^\intercal \mathbf{w}\]
\end{itemize}


\newpage


%\subsubsection*{3. Gaussian Distributions (10 points)}
%Let $X\sim N(0,1)$ and $Y=WX$, where $p(W=-1)=p(W=1)=0.5$. It is clear that $X$ and $Y$ are not independent since $Y$ is a function of $X$. 
%\begin{itemize}
%	\item Show $Y\sim N(0,1)$
%	\item Show $cov[X,Y]=0$. hint: $cov[X,Y]=E[XY]-E[X]E[Y]$ and $E[XY]=E[E[XY|W]]$
%\end{itemize}
%Therefore, $X$ and $Y$ are uncorrelated and Gaussian, but they are dependent. Why?
\subsubsection*{3. Gradient descent for fitting GMM (15 points)}
Consider the Gaussian mixture model
\[p(\mathbf{x}|\theta)=\sum_k \pi_{k=1}^K \mathcal{N}(\mathbf{x}|\mu_k,\Sigma_k)\]
Define the log likelihood as
\[ l(\theta) = \sum_{n=1}^N \log p(\mathbf{x}_n|\theta)
\]
Denote the posterior responsibility that cluster $k$ has for datapoint $n$ as follows:
\[
r_{nk}:=p(z_n=k|\mathbf{x}_n,\theta) = \frac{\pi_k\mathcal{N}(\mathbf{x}_n|\mu_k,\Sigma_k)}{\sum_{k'}\pi_{k'}\mathcal{N}(\mathbf{x}_n|\mu_{k'},\Sigma_k{k'})}
\]

\begin{itemize}
	
	\item Show that the gradient of the log-likelihood wrt $\mu_k$ is
	\[ \frac{d}{d\mu_k}l(\theta) = \sum_n r_{nk}\Sigma_k^{-1}(\mathbf{x}_n-\mu_k)
	\]
    \item Derive the gradient of the log-likelihood wrt $\pi_k$ without considering any constraint on $\pi_k$. (bonus: with constraint $\sum_k\pi_k=1$.)
    \item Derive the gradient of the log-likelihood wrt $\Sigma_k$ without considering any constraint on $\Sigma_k$. (bonus: with constraint $\Sigma_k$ be a symmetric positive definite matrix.) 
	
\end{itemize}


\newpage

\nocite{*} 
\bibliographystyle{apalike}
\bibliography{reference.bib}


\end{document}